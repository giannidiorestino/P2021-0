\documentclass[12pt,a4paper]{amsart}
%\usepackage[utf8]{inputenc}

\usepackage{amsmath,amssymb}
\swapnumbers
\usepackage{bbm}
%\usepackage{bibentry}
\usepackage{bm}
\usepackage{fullpage}
\usepackage{macros}
\usepackage{mathabx}
\usepackage{mathtools}

\DeclarePairedDelimiter\ceil{\lceil}{\rceil}
\DeclarePairedDelimiter\floor{\lfloor}{\rfloor}

\usepackage{cleveref}

%\newcommand{\Dspace}[2]{\mathbb D^{#1,#2}}
\newcommand{\Image}{\operatorname{image}}
\newcommand{\MRof}[2]{\operatorname{Mat}\left(#1\times#2\right)}
\newcommand{\Mof}[1]{\operatorname{Mat}\left(#1\times#1\right)}
%\newcommand{\OU}{\mathcal L}
%\newcommand{\Oof}[1]{\operatorname{O}(#1)}
%\newcommand{\Sfunctions}{\mathcal S}
%\newcommand{\bu}{\bm u}
%\newcommand{\by}{\bm y}
%\newcommand{\bz}{\bm z}
\newcommand{\calF}{\mathcal F}
\newcommand{\gaussian}[3]{\operatorname{N}_{#1}\left(#2,#3\right)}
\newcommand{\one}{\bm 1}
%\newcommand{\ppdiag}[1]{\diag_{++}\left(#1\right)}
%\renewcommand{\S}{\mathbb S}

\theoremstyle{plain}
\newtheorem{theorem}{Theorem}%[section]
\newtheorem{definition}[theorem]{Definition}%[section]
\newtheorem{proposition}[theorem]{Proposition}
\newtheorem{npar}{}%[section]

\theoremstyle{definition}

\theoremstyle{remark}
\newtheorem{remark}[theorem]{Remark}
\newtheorem{example}[theorem]{Example}
\newtheorem{exercise}[theorem]{Exercise}


\title{Probability 2021 \\ Part 0 \\ Recap of measure theory
}
\author[G. Pistone]{Giovanni Pistone}
\address{de Castro Statistics, Collegio Carlo Alberto}
\email{giovanni.pistone@carloalberto.org}
\urladdr{https://www.giannidiorestino.it/}
\date{DRAFT \today}

\begin{document}
\maketitle
\tableofcontents

This chapter offers a review of basic results in Measure Theory whose knowledge I consider useful even to those readers which
do not need a detailed knowledge of the general mathematical
theory. It is convenient to consult regularly reliable classical
textbooks when the need arises. I suggest consulting:
\cite{rudin:1987-3rd} for Analysis and Measure Theory,
\cite{jacod|protter:2003} for Probability with martingales,
\cite{malliavin:1995} for Probability with Fourier Analysis and
Gaussian Analysis, \cite{ross:2019introduction12} for a non-technical first introduction to probability models (Markov Chains,
Poisson Process, Continuous-Time Markov Chains, Renewal Theory,
Queueing Theory, Reliability Theory, Brownian Motion, Simulation,
Coupling), and \cite{efron|hastie:2016} for modern applications to
Statistics (Lasso, Random Forests, Neural Networks, Deep Learning,
Support-Vector Machines, Kernel Methods, Empirical Bayes). The SEP entry \cite{sep-probability-interpret} is a good starting point for the Philosophy of Probability.

\section{$\sigma$-algebra, measure, and probability}
\label{sec:probability}

\begin{definition}[Measure space]
The triple $(\Omega,\mathcal A,\mu)$ is a \emph{measure space} if
$\Omega$ is a set and the following holds.
\begin{enumerate}
\item $\mathcal A$ is a \emph{$\sigma$-algebra}, that is, $\mathcal A
  \subset 2^\Omega$, and
  \begin{enumerate}
  \item $\emptyset, \Omega \in \mathcal A$,
  \item $A \in \mathcal A$ implies $A^{\text c} \in \mathcal A$,
  \item
    $\cup_{i=1}^{\infty} A_i, \cap_{i=1}^{\infty} A_i \in \mathcal A$
    if $A_i \in \mathcal A$ for all $i \in \naturals$.
  \end{enumerate}
  \item $\mu$ is a \emph{$\sigma$-finite measure}, briefly, a
    \emph{measure}, that is, $\mu \colon \mathcal A \to [0,\infty]$ satisfies
  \begin{enumerate} 
  \item $\mu(\emptyset) = 0$,
  \item \label{item:measure-sigma}
    $\mu\left(\cup_{i=1 }^\infty A_i\right) = \sum_{i=1}^\infty
    \mu(A_i)$ whenever $(A_i)$ is a disjoint sequence in $\mathcal
    A$. Equivalently,
    $\mu(A \cup B) + \mu(A \cap B) = \mu(A) + \mu(B)$ and
    $B_i \uparrow B$ implies $\mu(B_i) \uparrow \mu(B)$.
  \item There is an increasing sequence $(\Omega_k)$ in $\mathcal A$
    such that $\Omega = \cup_{k=1}^\infty \Omega_k$ and
    $\mu(\Omega_k) < \infty$ for all $k$.
  \end{enumerate}
\end{enumerate}
\end{definition}
See \cite[Cap. 1]{rudin:1987-3rd}, or any other textbook.
\begin{remark}\ 
\begin{itemize}
\item The couple $(\Omega,\mathcal A)$ is a \emph{measurable space} and
$\mu$ is a measure on that measurable space. 
\item  Notice the up-arrow in
\cref{item:measure-sigma}. The continuity on $B_i \downarrow B$
requires $\mu(B_1) < \infty$. 
\item  A mesure $\mu$ is finite if
$\mu(\Omega) < \infty$. A finite measure is a probability measure if
$\mu(\Omega) =1$.
\item The set of all measures is a \emph{convex cone}, that is, if
  $\mu$ and $\nu$ are measures, then $(1-\lambda)\mu+\lambda \nu$,
  $0 \leq \lambda \leq 1$, and $\alpha \mu$, $\alpha \geq 0$, are
  measures. The set of all probability measures is a convex set. In
  general, a subset $C$ of a vector space $V$ is \emph{convex} if
  $x,y \in C$ implies $(1-\lambda)x + \lambda y \in C$ for all
  $0 \leq \lambda \leq 1$. It is a \emph{cone} if for all $x \in C$ it
  holds $\lambda x \in C$ for all $\lambda \geq 0$. In the case of
  measures, the definition of the enclosing vector space is not
  obvious.
\end{itemize}
\end{remark}

\begin{definition}[Generation of a $\sigma$-algebra]\ 
\begin{enumerate} 
  \item The family 
  $\mathcal C \subset \mathcal A$ is a \emph{generating family} if
  $\mathcal A$ is the smallest $\sigma$-algebra that contains
  $\mathcal C$, $\mathcal A = \sigma(\mathcal C)$.
\item If $S,\mathcal O)$ is a topological space, the $\sigma$-algebra
  generated by $\mathcal O$ is its \emph{Borel $\sigma$-algebra}.
\item A family $\mathcal C \subset \mathcal A$ is a \emph{Dynkin
    system} if
  \begin{enumerate}
  \item $A \setminus B \in \mathcal C$ whenever $A \supset B$ and $A,B
    \in \mathcal C$,
  \item If $(B_i)$ is an increasing sequence in $\mathcal C$, then
    $\cup_i B_i \in \mathcal C$.
  \end{enumerate}
\item A family $\mathcal C \subset \mathcal A$ is a
  \emph{$\pi$-system} if $\Omega \in \mathcal C$ and it is closed
  under intersection.
  \end{enumerate}
\end{definition}

\begin{theorem}[Equality of measures]\ 
  \begin{enumerate} 
  \item  \label{item:dynkin} $\mathcal C$ is \emph{both} a Dynkin system and a
    $\pi$-system if, and only if, it is a $\sigma$-algebra.
  \item \label{item:equality} If $\mu_1$ and $\mu_2$ are two measures on
    $(\Omega,\mathcal A)$ such that $\mu_1(C) = \mu_2(C)$ for all
    $C \in \mathcal C$, and $\mathcal C$ is a $\pi$-system and
    generates $\mathcal A$, then $\mu_1 = \mu_2$.
\end{enumerate}
\end{theorem}

See \cite[Ch. 6]{jacod|protter:2003} or \cite[Sec. 1.1]{malliavin:1995}.

\begin{theorem}[Fatou for sets]
If $(\Omega,\mathcal A,\mu)$ is a measure space and $(A_i)$ is a
sequence of measurable sets, then
\begin{equation*}
  \mu(\liminf_{i\to\infty} A_i) \leq \liminf_{i\to\infty} \mu(A_i) \ .
\end{equation*}
If, moreover, $\mu(\cup_{j > i} A_i) < \infty$ for some $j$, then
\begin{equation*}
  \limsup_{i\to\infty} \mu(A_i) \leq \mu(\limsup_{i\to\infty} A_i) \ .
\end{equation*}
\end{theorem}

\begin{proof} The proof is simple and the result is important. As
  $\mu$ is continous on increasing sequences,
  \begin{equation*}
\mu(\liminf_{i\to\infty} A_i) = \mu(\cup_i \cap_{j\ge i} A_) = \sup_i
\mu(\cap_{j \geq i} A_j) \leq \sup_i \inf_{j > i} \mu(A_j) =
\liminf_{i\to\infty} \mu(A_i) \ .   
  \end{equation*}
\end{proof}

\begin{theorem}[Product measure] Given measure spaces
  $(\Omega_1,\mathcal A_1,\mu_1)$ and $(\Omega_2,\mathcal A_2,\mu_2)$,
  there exixts a unique \emph{product space} $(\Omega_1 \times
  \Omega_2, \mathcal A_1 \otimes \mathcal A_2, \mu_1 \otimes \mu_2)$, where:
  \begin{enumerate}
  \item $\Omega = \Omega_1 \times \Omega _2$ is the Cartesian product.
  \item $\mathcal A = \mathcal A_1 \otimes \mathcal A_2$ is the
    $\sigma$-algebra generated by the rectangles $A_1 \times A_2$,
    $A_1 \in \mathcal A_1$ and $A_2 \in \mathcal A_2$. 
  \item $\mu = \mu_1 \otimes \mu_2$ is characterised by $\mu(A_1
    \times A_2) = \mu_1(A_1) \mu_2(A_2)$.
  \end{enumerate}
For each $A \in \mathcal A$ and each $\omega_2 \in \Omega_2$, the
section $A_{\omega_2} = \setof{\omega_1}{(\omega_2,\omega_2) \in A}$
belongs to $\mathcal A_1$.
\end{theorem}

\begin{proof}
  The proof uses integration, see below. Let us check the last
  statement. Let $\overline {\mathcal A}$ be the family of all subsets of
  $\Omega$ such that the section is measurable. This family is a
  $\sigma$-algebra and contains all rectangles, hence $\overline {\mathcal A}
  \supset \mathcal A$.
\end{proof}

\begin{theorem}[Counting measure] Let $\Omega$ be denumerable. There
  exists a unique measure $\mu$ on $\Omega,2^\Omega$ such that
  $\mu(\omega) = 1$, $\omega \in \Omega$.
\end{theorem}

\begin{theorem}[Lebesgue measure] Let $\mathcal B$ be the
  $\sigma$-algebra generated in $\reals$ by the intervals of the form
  $]-\infty,a]$, $a\in\reals$. There exists a unique measure $\mu$ on
  $(\reals,\mathcal B)$ such that $\mu(]a,b])=b-a$. Such a measure has
  a unique extension to the $\sigma$-algebra $\mathcal L$ generated by
  $\mathcal B$ and all the null-sets. The Lebesgue measure of
  $\reals^n$ is the (completed) product of the factor Lebesgue measures.
\end{theorem}
See \cite[Th. 2.20]{rudin:1987-3rd}.

\section{Measurable functions}
\label{sec:measurable-functions}

\begin{definition}[Measurable function]
  Let be given measurable spaces $(\Omega,\mathcal A)$ and
  $(S,\mathcal S)$. A function $f \colon \Omega \to S$ is \emph{measurable}
  (or a \emph{random variable}) if $f^{-1} \colon \mathcal S \to
  \mathcal A$. If $\mu$ is a measure on $(\Omega,\mathcal A)$, then
  $f_\# = \mu \circ f^{-1}$ is a measure on $(S,\mathcal S)$.
\end{definition}

\begin{remark}\ 
\begin{itemize}
\item $f_\#$ has many names: \emph{image}, \emph{push-forward},   \emph{distribution}.
\item A measurable function on a probability space is called a
  \emph{random variable}.
\end{itemize}
\end{remark}

\begin{theorem}[Checking that a function is
  measurable]\label{th:checkmeasurable}\
  \begin{enumerate}
  \item If $\mathcal D$ is a generating family for $\mathcal S$ and
    $f^{-1}(\mathcal D) \subset \mathcal A$, then $f$ is measurable.
  \item Given $f_1 \colon \Omega \to S_1$ and $f_2 \colon \Omega \to
    S_2$, and measurable spaces $(\Omega,\mathcal A)$, $(S_1,\mathcal
    S_1)$, and $(S_2,\mathcal S_2)$, both $f_1$ and $f_2$ are
    measurable if, and only if, the random variable $(f_1,f_2) \colon
    \Omega \to (S_1 \times S_2)$ is measurable for $\mathcal S_1
    \otimes \mathcal S_2$.
  \item If $f \colon \Omega \to S_1$ is measurable from
    $(\Omega,\mathcal A)$ to $(S,\mathcal S_1)$ and
    $g \colon S_1 \to S_2$ is measurable from $(S_1,\mathcal S_1)$ to
    $(S_2,\mathcal S_2)$, then $g \circ f \colon \Omega \to S_2$ is
    measurable from $(\Omega,\mathcal A)$ to $(S_2,\mathcal
    S_2)$.
  \item A real function $f \colon \Omega \to \reals$ is measurable
    from $(\Omega,\mathcal A)$ to $(\reals,\mathcal B)$, $\mathcal B$
    Borel $\sigma$-algebra of $\reals$ if, and only is,
    $\setof{\omega}{f(\omega) \leq a} \in \mathcal A$ for all
    $a \in \reals$.
  \item A real function $f \colon \Omega \to \reals$ is measurable
    from $(\Omega,\mathcal A)$ to $(\reals,\mathcal B)$, $\mathcal B$
    Borel $\sigma$-algebra of $\reals$ if, and only is, both the
    positive part $f^+$ and the negative part $f^-$ are measurable.
  \item The set of real measurable functions $\mathcal L$ is a real algebra with
    unit. $\mathcal L$ is stable for sequential $\inf$, $\sup$, $\liminf$,
    $\limsup$, $\lim$.
  \item \label{item:upsimple} A non-negative function $f$ is measurable from
    $(\Omega,\mathcal A)$ to $(\reals,\mathcal B)$ if, and only if,
    there exists an increasing sequence $g_j$ of simple measurable
    functions
    \begin{equation*}
      g_j(\omega) = \sum_{x \in \operatorname{Image} g_j} x (\omega : f_j(\omega) = x)
    \end{equation*}
    such that $f_j(\omega) \uparrow f(\omega)$.
  \end{enumerate}
\end{theorem}
\section{Integral}
\label{sec:integrals}

\begin{theorem}[Integral of a positive measurable function]\ 
\begin{enumerate}
% \item Let $f$ be a non-negative measurable function on the measurable space
% $(\Omega,\mathcal A)$, $f \in \mathcal L_+(\mathcal A)$ and let $(g_j)$ be a sequence of simple
% functions increasing point-wise to $f$, see
% \cref{th:checkmeasurable}\eqref{item:upsimple}.
\item Le be given a measure $\mu$ on $(\Omega,\mathcal A)$.  For each
  simple function $g$ define
  \begin{equation*}
    \int g \ d\mu = \sum_{x \in
      \operatorname{Image} g} x \mu\setof{\omega}{f_j(\omega) = x}
    \in \reals \ .  
  \end{equation*}
\item For each non-negative measurable $f$ define
  \begin{equation*}
    \int f \ d\mu = \sup \setof{\int g \ d\mu}{\text{$g$ is simple and
      $g \leq f$}} \ .
  \end{equation*}
\item If $f, g \in \mathcal L_+$ then
  \begin{equation*}
    f \leq g \quad \Rightarrow \quad \int f \ d\mu \leq \int g \ d\mu \ .
  \end{equation*}
\item Given constants $\alpha,\beta \geq 0$ and functions $f,g \in
  \mathcal L_+(\mathcal A)$, it holds
  \begin{equation*}
    \int (\alpha f + \beta g) \ d\mu = \alpha \int f \ d\mu + \beta
    \int g \ d\mu
  \end{equation*}
in the algebra of $[0,\infty]$. 
\item (\emph{$\sigma$-additivity})For each sequence $(f_j)$ of non-negative measurable function such that $\sum_{j=1}^\infty f_j(\omega) = f(\omega) < \infty$, 
  it holds
  \begin{equation*}
    \int \sum_{j=1}^\infty f_j \ d\mu = \sum_{j=1}^\infty  \int f_j \ d\mu \ .
  \end{equation*}
\item (\emph{Beppo Levi}) For each non-decreasing sequence $(f_j)$ of non-negative measurable function
  such that $\int \sup_{j} f_j \ d\mu < \infty$, then $\sup_j f_j = f
  \in \mathcal L_+$ and $\int f \ d\mu = sup_j \int f_j \ d\mu <
  \infty$. Moreover, $\lim_{j \to \infty} \inf (f - f_j) \ d\mu = 0$.  
\item (\emph{Fatou}) For each sequence $(f_j)$ of non-negative measurable function
  such that $\liminf_{j\to\infty} f_j = f \in \mathcal L$, it  holds,
  \begin{equation*}
    \int \liminf_{j \to \infty} f_j \ d\mu \leq \liminf_{j \to \infty}
    \int f_j \ d\mu \ .
  \end{equation*}
\end{enumerate}
\end{theorem}

\begin{definition}
  If $f$ is a real measurable function on $(\Omega,\mathcal A)$, then
  $f = f^+ - f^-$. $f^+$ and $f^-$ are non-negative measurable
  functions. Given a measure $\mu$, If $\int f^- \ d\mu < \infty$,
  then we say that $f$ is \emph{quasi-integrable} define
  \begin{equation*}
    \int f \ d\mu = \int f^+ \ d\mu - \int f^- \ d\mu \ .
  \end{equation*}
If $\int \avalof f \ d\mu < \infty$, then $f$ is \emph{integrable}, $f
\in \mathcal L^1(\mu)$.
\end{definition}

\begin{theorem}[Integrable functions]\
  \begin{enumerate}
  \item The set $\mathcal L^1(\mu)$ of integrable real functions is a
    vector space.
  \item The mapping
    \begin{equation*}
      \mathcal L^1(\mu) \ni f \mapsto \int f \ d\mu \in \reals
    \end{equation*}
is linear and
\begin{equation*}
  \avalof{\int f \ d\mu} \leq \int \avalof f \ d\mu \ .
\end{equation*}
\item (\emph{Lebesgue}) Let $(f_j)$ be a sequence in $\mathcal
  L^1(\mu)$. Assume that $\lim_j f_j = f$ pointwise and there exists a
  $g \in \mathcal L^1(\mu)$ such that $\avalof{f_j} \leq g$. Then, $f
  \in \mathcal L^1(\mu)$ and $\lim_{j\to\infty} \int \avalof {f - f_j}
  \ d\mu = 0$. 
  \end{enumerate}

\end{theorem}
\begin{theorem}[Jensen's inequality]
  Let $f$ be integrable. For each probability measure $\mu$ and
  each convex function $\Phi \colon \reals \to \reals$ it holds
  \begin{equation*}
    \Phi\left(\int f \ d\mu\right) \leq \int \Phi\circ f \ d\mu \ .
  \end{equation*}
  \begin{proof}
    If $\Phi$ is convex, for each $x_0$ there exists a $\beta$ such that
    \begin{equation*}
      \Phi(x) - \Phi(x_0) \geq \beta(x - x_0) \quad x \in \reals \ .
    \end{equation*}
In particular, if $x = f(\omega)$ and $x_0 = \int f \ d\mu$, then
\begin{equation*}
  \Phi\left(f(\omega)\right) \geq \Phi\left( \int f \ d\mu\right) +
  \beta\left(f(\omega) - \int f \ d\mu\right) \ , 
\end{equation*}
and the inequality follows from the monotonicity of the integral. See
\cite[Th. 3.3]{rudin:1987-3rd}.
  \end{proof}
\end{theorem}

\begin{remark}(Almost everywere) If $\mu\setof{\omega \in
    \Omega}{f(\omega) = g(\omega)} = 0$, we say that $f$ and $g$ are
  equal almost everywere. In such a case, $\int f \ d\mu = \int g \
  d\mu$. For this reason, we consider the integral defined of
  functions that are not defined on a measurable set of measure 0.  
\end{remark}

\begin{theorem}[Product measure]\
  \begin{enumerate}
  \item (\emph{Existence)} Given two measure spaces $(\Omega_1,\mathcal A_1,\mu_1)$ and
    $(\Omega_2,\mathcal A_2,\mu_2)$, there exixst a unique 
    measure $\mu_1 \otimes \mu_2$ on the product measurable space
    $(\Omega_1 \times \Omega_2, \mathcal A_1 \otimes A_2)$ such that
    \begin{equation*}
      \mu_1 \otimes \mu_2 (A_1 \times A_2) = \mu_1(A_1) \mu_2(A_2) \ .
    \end{equation*}
\item (\emph{Tonelli}) If $f \in \mathcal L_+(\mathcal A_1 \otimes \mathcal A_2)$, then
  \begin{multline*}
    \int f \ d\mu_1\otimes\mu_2 = \\ \int_{\Omega_1} \left(\int_{\Omega_2}
    f(\omega_1,\omega_2)  \ \mu_2(d\omega_2)\right) \ \mu_1(d\omega_1)
  = \int_{\Omega_2} \left(\int_{\Omega_1}
    f(\omega_1,\omega_2)  \ \mu_1(d\omega_1)\right) \ \mu_2(d\omega_2)
  \ .
  \end{multline*}
\item (\emph{Fubini}) If $f \in \mathcal L^1(\mu_1 \otimes \mu_2)$, then
  \begin{multline*}
    \int f \ d\mu_1\otimes\mu_2 = \\ \int_{\Omega_1} \left(\int_{\Omega_2}
    f(\omega_1,\omega_2)  \ \mu_2(d\omega_2)\right) \ \mu_1(d\omega_1)
  = \int_{\Omega_2} \left(\int_{\Omega_1}
    f(\omega_1,\omega_2)  \ \mu_1(d\omega_1)\right) \ \mu_2(d\omega_2)
  \ .
  \end{multline*}
  \end{enumerate}
\end{theorem}

\begin{proof}
  The three statements above are actually variation of a single
  constructive argument. Without restriction of generality, assume
  that the meaasures are both finite. Then, one checks that for each
  $C \in \mathcal A_1 \otimes A_2$ the function
  $\omega_2 \mapsto \int_{\Omega_1} (\omega_1 \colon
  (\omega_1,\omega_2) \in C) \ \nu_1(d\omega_1)$ is well defined,
  measurable, positive, finite. The last step is the proof that the
  iterated integral provides the required product measure. See
  \cite[Sec. 8.1-9]{rudin:1987-3rd}.
\end{proof}

\begin{theorem}[Change of variable] Let $\phi$ be a measurable
  function from $(\Omega,\mathcal A)$ to $(S,\mathcal S)$. Let $\nu$
  be the push-formard of the measure $\mu$, $\nu = \phi_{\#} \mu = \mu
  \circ \phi^{-1}$. Then, for each non-negative measurable $f \colon S
  \to \reals$,
  \begin{equation*}
    \int f \ d\nu = \int f \circ \phi \ d\mu \ .
  \end{equation*}
  As a consequence, for each integrable $f \in \mathcal L^1(\nu)$, it
  holds $f \circ \phi \in \mathcal L^1(\mu)$ and the equality above
  holds.  
\end{theorem}

\begin{exercise} Prove the Fatou theorem both for sets and for non-negative
    functions.
  \end{exercise}

  \begin{exercise} The following exercise reviews a fundamental
    example of stochastic process in its simplest form, that is the
    construction of independent trial (flipping of a fair coin) due
    originally to E. Borel (1909). The sistematic study of the
    ``simplest non-trivial example'' is the most important tool in the
    undestanding of a general mathematical theory.

    Consider the sample space of \emph{Bernoulli trials}, that is,
    \begin{equation*}
      \Omega = \set{0,1}^\naturals \ni \omega = (x_1,x_2, \dots) \ ,
      \quad x_j = 0,1 \ , \quad j=1,2,\dots
    \end{equation*}
    The $j$-th trial in the infinite sequnce of binary trials is the
    mapping $X_j : \Omega \to S=\set{0,1}$, $X_j(\omega) = x_j$ if
    $x_j$ is the $j$-th element of $\omega$. Let $\mathcal F$ be the
    $\sigma$-algebra generated by $X_j$, $j \in \naturals$ and
    $\mathcal F_n$ the $\sigma$-algebra generated by $X_1,\dots,X_n$,
    $n=1,2,\dots$. Consider the random varibles
    \begin{equation*}
      F_n = \frac 1 t \sum_{j=1}^dnt X_j \ , \quad F_t \colon \Omega \to
      \setof{\frac k n}{\sum_{j=1}^n X_j = k} \subset [0,1] \ .
    \end{equation*}
    The basis of the \emph{Strong Law of Large Numbers} is the remark
    that the set
\begin{equation*}
  G = \setof{\omega}{\lim_{n\to\infty} F_n(\omega) = \frac 12}
\end{equation*}
is in $\mathcal F$. The second step is to provide a probability
measure $\mu$ on $(\Omega,\mathcal F)$ such that $\mu(G) = 1$. The
proof of the measurability of $G$ follows from
\begin{equation*}
  G(m,k) = \setof{\omega \in \Omega}{\avalof{F_m(\omega) - \frac12 } \leq
    \frac 1k}
\end{equation*}
and
\begin{equation*}
  (\omega \in G) \quad \Leftrightarrow \quad \forall k \exists n \forall
  m \geq n \ (\omega \in G(m,k)) \ . 
\end{equation*}


Fix $n \in \naturals$ and consider the mapping
$(X_1,\dots,X_n)$ defined by
\begin{equation*}
  \Omega \ni \omega \mapsto (X_1(\omega),\dots,X_n(\omega)) \in
  S^{n} = \underbracket{S \times \cdots \times S}_{\text{$n$ times}} \ .
\end{equation*}
Note that $\mathcal S = \set{\emptyset, \set 0, \set 1, S}$ is a
$\sigma$-algebra on $S$ and that $\mathcal S^{\otimes n} =
2^{S^n}$. Moreover, $\mathcal F_n = (X_1,\dots,X_n)^{-1} \mathcal
S^{\otimes n}$. It follows that each $A \in \mathcal F_n$ is of the form
\begin{equation*}
  A = \setof{\omega}{(X_1(\omega),\dots,X_n(\omega)) \in B} \ , \quad
    B \in \mathcal S^{\otimes n} \ ,
  \end{equation*}
  and each real random variable $f \in \mathcal L(\mathcal
  F_n)$ is of the form
  \begin{equation*}
    f(\omega) = g(X_1(\omega),\dots,X_n(\omega)) \ \quad g \colon S^n
    \to \reals \ .
  \end{equation*}

Let $\nu_n$ be the uniform probability measure on
$S^n,\mathcal S^{\otimes n}$, that is $\nu_n(x_1,\dots,x_n) = 2^{-n}$. Then
\begin{equation*}
  \mu_n(A) = \nu_n(B) = \#(B) 2^{-n}
\end{equation*}
defines a probability measure on $(\Omega, \mathcal F_n)$. The
sequence $(\mu_n)$ is consistent, that is, each $\mu_n$ is
the restriction of $\mu_{n+1}$ to $\mathcal F_n$.

The \emph{shift} is the mapping $\tau \colon \Omega \to \Omega$
defined by 
\begin{equation*}
  \tau(x_1,x_2,\dots) = (x_2,x_3,\dots) \ , \quad \omega =
  (x_1,x_2,\dots) \in \Omega \ .
\end{equation*}
It holds $X_j\circ \tau = X_{j+1}$ and the shift is a measurable
mapping from $(\Omega, \mathcal F)$ to itself. The mapping $S \colon
\Omega \to [0,1]$ is well defined by
\begin{equation*}
  S(\omega) = \sum_{j=1}^\infty X_j(\omega) 2^{-j} \ .
\end{equation*}
$S$ is not a bijection because $S(1,0,0,\dots) = S(0,1,1,\dots)$. It holds
\begin{equation*}
  S\circ \tau (\omega) = \sum_{j=1}^\infty X_{j+1}(\omega) 2^{-j} = 2
  S(\omega) - X_1(\omega) \ .
\end{equation*}

It is possible to define a function $S^+ \colon [0,1[ \to \Omega$ with
the following properties: 1) $S^+$ is measurable from $([0,1[ ,
\mathcal B)$ to $(\Omega,\mathcal F)$; 2) $S^+$ is a pseudo inverse of
$S$, that is $S \circ S^+(\theta) = \theta$, $\theta \in [0,1[$; 3) the push-forward of the Lebesgue measure is a measure $\mu$
on $(\Omega,\mathcal F)$ whose restriction to $\mathcal F_n$ is
$\mu_n$, $n=1,2,\dots$. One option is to consider the mapping
\begin{equation*}
  \reals_+ \ni \theta \mapsto (-1)^{\floor*{\theta}}
  =
  \begin{cases}
   +1 & \theta \in [0,1[ \cup [2,3[ \cup \cdots \\
   -1 & \theta \in [1,2[ \cup [3,4[ \cup \cdots
  \end{cases}
\end{equation*}
The group $\set{+1,-1;\times}$ is mapped to the group $\set{0,1;+_2}$
by $x = \frac12(1-y)$, with inverse $y = 1-2x$, so that
\begin{equation*}
 X \colon  \theta \mapsto \frac12\left(1 - (-1)^{\floor \theta}\right)
  =
  \begin{cases}
   0 & \theta \in [0,1[ \cup [2,3[ \cup \cdots \\
   1 & \theta \in [1,2[ \cup [3,4[ \cup \cdots
  \end{cases}
\end{equation*}
Now, define $S^+ \colon [0,1[ \to \Omega$ by  
\begin{equation*}
  S^+(\theta) = \left(X(2^1 \theta), (X(2^2 \theta), \dots \right) \ .
\end{equation*}
The function $S\circ S^+$ is right-continuous and for each 
\end{exercise}

\bibliographystyle{amsplain}
\bibliography{tutto}

\end{document}


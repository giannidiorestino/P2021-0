\documentclass[12pt,a4paper]{amsart}
%\usepackage[utf8]{inputenc}

\usepackage{amsmath,amssymb}
\usepackage{bbm}
\usepackage{bibentry}
\usepackage{bm}
\usepackage{cleveref}
\usepackage{fullpage}
\usepackage{macros}
\usepackage{mathabx}
\usepackage{mathtools}

\newcommand{\Dspace}[2]{\mathbb D^{#1,#2}}
\newcommand{\Image}{\operatorname{image}}
\newcommand{\MRof}[2]{\operatorname{Mat}\left(#1\times#2\right)}
\newcommand{\Mof}[1]{\operatorname{Mat}\left(#1\times#1\right)}
\newcommand{\OU}{\mathcal L}
\newcommand{\Oof}[1]{\operatorname{O}(#1)}
\newcommand{\Sfunctions}{\mathcal S}
\newcommand{\bu}{\bm u}
\newcommand{\by}{\bm y}
\newcommand{\bz}{\bm z}
\newcommand{\calF}{\mathcal F}
\newcommand{\gaussian}[3]{\operatorname{N}_{#1}\left(#2,#3\right)}
\newcommand{\one}{\bm 1}
\newcommand{\ppdiag}[1]{\diag_{++}\left(#1\right)}
\renewcommand{\S}{\mathbb S}

\theoremstyle{plain}% default
\newtheorem{theorem}{Theorem}%[section]
\newtheorem{proposition}[theorem]{Proposition}
\newtheorem{npar}{}%[section]
\theoremstyle{definition}
\newtheorem{definition}[theorem]{Definition}%[section]
\theoremstyle{remark}
\newtheorem{remark}[theorem]{Remark}
\newtheorem{example}[theorem]{Example}

\title{Probability 2021 \\ Part 0
}
\author[G. Pistone]{Giovanni Pistone}
\address{de Castro Statistics, Collegio Carlo Alberto}
\email{giovanni.pistone@carloalberto.org}
\urladdr{https://www.giannidiorestino.it/}
\date{DRAFT \today}

\begin{document}
\maketitle
\tableofcontents

This chapter is intended to be a review of basic results in Measure
Theory whose knowledge I consider useful even to those readers which
do not need a detailed knowledge of the general mathematical
theory. It is convenient to regularly consult reliable classical
textbooks when need arises. I suggest to consult:
\cite{rudin:1987-3rd} for Analysis and Measure Theory,
\cite{jacod|protter:2003} for Probability with martingales,
\cite{malliavin:1995} for Probability with Fourier Analysis and
Gaussian Analysis, \cite{ross:2019introduction12} for a first
non-technical introduction to probability models (Markov Chains,
Poisson Process, Continuous-Time Markov Chains, Renewal Theory,
Queueing Theory, Reliability Theory, Brownian Motion, Simulation,
Coupling), and \cite{efron|hastie:2016} for modern applications to
Statistics (Lasso, Random Forests, Neural Networks, Deep Learning,
Support-Vector Machines, Kernel Methods, Empirical Bayes).

\section{Measure and probability}
\label{sec:probability}

\begin{definition}
The triple $(\Omega,\mathcal A,\mu)$ is a \emph{measure space} if
$\Omega$ is a set and the following holds.
\begin{enumerate}
\item $\mathcal A$ is a \emph{$\sigma$-algebra}, that is, $\mathcal A
  \subset 2^\Omega$, and
  \begin{enumerate}
  \item $\emptyset, \Omega \in \mathcal A$,
  \item $A \in \mathcal A$ implies $A^{\text c} \in \mathcal A$,
  \item
    $\cup_{i=1}^{\infty} A_i, \cap_{i=1}^{\infty} A_i \in \mathcal A$
    if $A_i \in \mathcal A$ for all $i \in \naturals$.
  \end{enumerate}
  \item $\mu$ is a \emph{$\sigma$-finite measure}, briefly, a
    \emph{measure}, that is, $\mu \colon \Omega \to [0,\infty]$ satisfies
  \begin{enumerate} 
  \item $\mu(\emptyset) = 0$,
  \item \label{item:measure-sigma}
    $\mu\left(\cup_{i=1 }^\infty A_i\right) = \sum_{i=1}^\infty
    \mu(A_i)$ whenever $(A_i)$ is a disjoint sequence in $\mathcal
    M$. Equivalently,
    $\mu(A \cup B) + \mu(A \cap B) = \mu(A) + \mu(B)$ and
    $B_i \uparrow B$ implies $\mu(B_i) \uparrow \mu(B)$.
  \item There is an increasing sequence $(\Omega_k)$ in $\mathcal A$
    such that $\Omega = \cup_{k=1}^\infty \Omega_k$ and
    $\mu(\Omega_k) < \infty$ for all $k$.
  \end{enumerate}
\end{enumerate}
\end{definition}
Notice the up-arrow in \cref{item:measure-sigma}. The continuity on
$B_i \downarrow B$ requires $\mu(B_1) < \infty$.  A mesure $\mu$ is
finite if $\mu(\Omega) < \infty$. A finite measure is a probability
measure if $\mu(\Omega) =1$. The couple $(\Omega,\mathcal A)$ is a
\emph{measurable space} and $\mu$ is a measure on that measurable
space. The set of all measures is closed for positive combinations.

\begin{definition}[Generation of a $\sigma$-algebra]\ \begin{enumerate}
  \item The family 
  $\mathcal C \subset \mathcal A$ is a \emph{generating family} if
  $\mathcal A$ is the smallest $\sigma$-algebra that contains
  $\mathcal C$, $\mathcal A = \sigma(\mathcal C)$.
\item A family $\mathcal C \subset \mathcal A$ is a \emph{Dynkin
    system} if
  \begin{enumerate}
  \item $\Omega \in \mathcal C$,
  \item $A \setminus B \in \mathcal C$ whenever $A \supset B$ and $A,B
    \in \mathcal C$,
  \item If $(B_i)$ is an increasing sequence in $\mathcal C$, then
    $\cup_i B_i \in \mathcal B$.
  \end{enumerate}
  \item A family $\mathcal C \subset \mathcal A$ is a
    \emph{$\pi$-system} if it is closed under intersection.  
  \end{enumerate}
\end{definition}

\begin{theorem}[Equality of measures]
  \begin{enumerate}\ 
  \item  \label{item:dynkin} $\mathcal C$ is \emph{both} a Dynkin system and a
    $\pi$-system if, and only if, it is a $\sigma$-algebra.
  \item \label{item:equality} If $\mu_1$ and $\mu_2$ are two measures on
    $(\Omega,\mathcal A)$ such that $\mu_1(C) = \mu_2(C)$ for all
    $C \in \mathcal C$, and $\mathcal C$ is a $\pi$-system and
    generates $\mathcal A$, then $\mu_1 = \mu_2$.
\end{enumerate}
\end{theorem}
See \cite[Ch. 6]{jacod|protter:2003} or \cite[Sec. 1.1]{malliavin:1995}.

\begin{theorem}[Fatou for sets]
If $(\Omega,\mathcal A,\mu)$ is a measure space and $(A_i)$ is a
sequence of measurable sets, then
\begin{equation*}
  \mu(\liminf_{i\to\infty} A_i) \leq \liminf_{i\to\infty} \mu(A_i) \ .
\end{equation*}
If, moreover, $\mu(\cup_{j > i} A_i) < \infty$ for some $j$, then
\begin{equation*}
  \limsup_{i\to\infty} \mu(A_i) \leq \mu(\limsup_{i\to\infty} A_i) \ .
\end{equation*}
\begin{proof} The proof is simple and the result is important. As
  $\mu$ is continous on increasing sequences,
  \begin{equation*}
\mu(\liminf_{i\to\infty} A_i) = \mu(\cup_i \cap_{j\ge i} A_) = \sup_i
\mu(\cap_{j \geq i} A_j) \leq \sup_i \inf_{j > i} \mu(A_j) =
\liminf_{i\to\infty} \mu(A_i) \ .   
  \end{equation*}
\end{proof}
\end{theorem}
\begin{theorem}[Product measure] Given measure spaces
  $(\Omega_1,\mathcal A_1,\mu_1)$ and $(\Omega_2,\mathcal A_2,\mu_2)$,
  there exixts a unique \emph{product space} $(\Omega_1 \times
  \Omega_2, \mathcal A_1 \otimes \mathcal A_2, \mu_1 \otimes \mu_2)$, where:
  \begin{enumerate}
  \item $\Omega = \Omega_1 \times \Omega _2$ is the Cartesian product.
  \item $\mathcal A = \mathcal A_1 \otimes \mathcal A_2$ is the
    $\sigma$-algebra generated by the rectangles $A_1 \times A_2$,
    $A_1 \in \mathcal A_1$ and $A_2 \in \mathcal A_2$. 
  \item $\mu = \mu_1 \otimes \mu_2$ is characterised by $\mu(A_1
    \times A_2) = \mu_1(A_1) \mu_2(A_2)$.
  \end{enumerate}
For each $A \in \mathcal A$ and each $\omega_2 \in \Omega_2$, the
section $A_{\omega_2} = \setof{\omega_1}{(\omega_2,\omega_2) \in A}$
belongs to $\mathcal A_1$.
\end{theorem}

\begin{proof}
  The proof uses integration, see below. Let us check the last
  statement. Let $\overline {\mathcal A}$ be the family of all subsets of
  $\Omega$ such that the section is measurable. This family is a
  $\sigma$-algebra and contains all rectangles, hence $\overline {\mathcal A}
  \supset \mathcal A$.
\end{proof}

\begin{theorem}[Counting measure] Let $\Omega$ be denumerable. There
  exists a unique measure $\mu$ on $\Omega,2^\Omega$ such that
  $\mu(\omega) = 1$, $\omega \in \Omega$.
\end{theorem}

\begin{theorem}[Lebesgue measure] Let $\mathcal B$ be the
  $\sigma$-algebra generated in $\reals$ by the intervals of the form
  $]-\infty,a]$, $a\in\reals$. There exists a unique measure $\mu$ on
  $(\reals,\mathcal B)$ such that $\mu(]a,b])=b-a$. Such a measure has
  a unique extension to the $\sigma$-algebra $\mathcal L$ generated by
  $\mathcal B$ and all the null-sets. The Lebesgue measure of
  $\reals^n$ is the (completed) product of the factor Lebesgue measures.
\end{theorem}
See \cite[Th. 2.20]{rudin:1987-3rd}.

\section{Measurable functions}
\label{sec:measurable-functions}

\begin{definition}
  Let be given measurable spaces $(\Omega,\mathcal A)$ and
  $(S,\mathcal S)$. A function $f \colon \Omega \to S$ is \emph{measurable}
  (or a \emph{random variable}) if $f^{-1} \colon \mathcal S \to
  \mathcal A$. If $\mu$ is a measure on $(\Omega,\mathcal A)$, then
  $f_\# = \mu \circ f^{-1}$ is a measure on $(S,\mathcal S)$. $f_\#$ has many names: \emph{image}, \emph{push-forward},
\emph{distribution}.
\end{definition}


\begin{theorem}[Tensorialization and composition of random variables]
\end{theorem}
\section{Integrals}
\label{sec:integrals}

\bibliographystyle{amsplain}
\bibliography{tutto}

\end{document}

\usepackage{macros}

\title{P2021-0}
\author{Giovanni Pistone}
\date{DRAFT \today}

\begin{document}

\maketitle

This is a list of relevant theorems with references.

\section{Probability}
\label{sec:probability}

\section{Integrals}
\label{sec:integrals}

\bibliographystyle{amsplain}
\bibliography{tutto}

\end{document}

\documentclass[12pt,a4paper]{amsart}
%\usepackage[utf8]{inputenc}

\usepackage{amsmath,amssymb}
\usepackage{bbm}
\usepackage{bibentry}
\usepackage{bm}
\usepackage{fullpage}
\usepackage{macros}
\usepackage{mathabx}
\usepackage{mathtools}

\newcommand{\Dspace}[2]{\mathbb D^{#1,#2}}
\newcommand{\Image}{\operatorname{image}}
\newcommand{\MRof}[2]{\operatorname{Mat}\left(#1\times#2\right)}
\newcommand{\Mof}[1]{\operatorname{Mat}\left(#1\times#1\right)}
\newcommand{\OU}{\mathcal L}
\newcommand{\Oof}[1]{\operatorname{O}(#1)}
\newcommand{\Sfunctions}{\mathcal S}
\newcommand{\bu}{\bm u}
\newcommand{\by}{\bm y}
\newcommand{\bz}{\bm z}
\newcommand{\calF}{\mathcal F}
\newcommand{\gaussian}[3]{\operatorname{N}_{#1}\left(#2,#3\right)}
\newcommand{\one}{\bm 1}
\newcommand{\ppdiag}[1]{\diag_{++}\left(#1\right)}
\renewcommand{\S}{\mathbb S}

\theoremstyle{plain}% default
\newtheorem{theorem}{Theorem}%[section]
\newtheorem{proposition}[theorem]{Proposition}
\newtheorem{npar}{}%[section]
\theoremstyle{definition}
\newtheorem{definition}{Definition}%[section]
\theoremstyle{remark}
\newtheorem{remark}{Remark}
\newtheorem{example}{Example}

\title{Probability 2021 \\ Part 0
}
\author[G. Pistone]{Giovanni Pistone}
\address{de Castro Statistics, Collegio Carlo Alberto}
\email{giovanni.pistone@carloalberto.org}
\urladdr{https://www.giannidiorestino.it/}
\date{DRAFT \today}

\begin{document}
\maketitle
\tableofcontents

This is a list of relevant theorems with references from \cite{rudin:1987-3rd}

\section{Measure and robability}
\label{sec:probability}

\begin{definition}
The triple $(\Omega,\mathcal M,\mu)$ is a \emph{measure space} if
$\Omega$ is a set and the following holds.
\begin{enumerate}
\item $\mathcal M$ is a \emph{$\sigma$-algebra}, that is, $\mathcal M
  \subset 2^\Omega$, and
  \begin{enumerate}
  \item $\emptyset, \Omega \in \mathcal M$,
  \item $A \in \mathcal M$ implies $A^{\text c} \in \mathcal M$,
  \item $\cup_{i=1}^{\infty} A_i \in \mathcal M$ if $A_i \in \mathcal
    M$ for all $i \in \naturals$.
  \end{enumerate}
  \item $\mu$ is a \emph{$\sigma$-finite measure}, briefly, a
    \emph{measure}, that is, $\mu \colon \Omega \to [0,\infty]$ satisfies
  \begin{enumerate} 
  \item $\mu(\emptyset) = 0$,
  \item $\mu\left(\cup_{i=1 }^\infty A_i\right) = \sum_{i=1}^\infty
    \mu(A_i)$ whenever $(A_i)$ is a disjoint sequence in $\mathcal M$.
  \item There is an increasing sequence $(\Omega_k)$ in $\mathcal M$
    such that $\Omega = \cup_{k=1}^\infty \Omega_k$ and
    $\mu(\Omega_k) < \infty$ for all $k$.
  \end{enumerate}
\end{enumerate}
\end{definition}

A mesure $\mu$ is finite if $\mu(\Omega) < \infty$. A finite measure
is a probability measure if $\mu(\Omega) =1$. The couple
$(\Omega,\mathcal M)$ is a \emph{measurable space} and $\mu$ is a
measure on that measurable space. The set of all measures is closed
for positive combinations.

\begin{theorem}[Equality of measures] Let $\mathcal C \subset \mathcal M$ be a
  generating family, that is, $\mathcal M$ is the smallest
  $\sigma$-algebra that contains $\mathcal C$, $\mathcal M =
  \sigma(\mathcal C)$. Assume that $\mathcal C$ is closed under
  intersection. If $\mu_1$ and $\mu_2$ are two measures on
  $(\Omega,\mathcal M)$ such that $\mu_1(C) = \mu_2(C)$ for all $C \in
  \mathcal C$, then $\mu_1 = \mu_2$. 
\end{theorem}

\begin{theorem}[Counting measure] Let $\Omega$ be denumerable. There
  exists a unique measure $\mu$ on $\Omega,2^\Omega$ such that
  $\mu(\omega) = 1$, $\omega in \Omega$.
\end{theorem}

\begin{theorem}[Lebesgue measure] Let $\mathcal B$ be the
  $\sigma$-algebra generated in $\reals$ by the intervals of the form
  $]-\infty,a]$, $a\in\reals$. There exists a unique measure $\mu$ on
  $(\reals,\mathcal B)$ such that $\mu(]a,b])=b-a$. Such a measure has
  a unique extension to the $\sigma$-algebra $\mathcal L$ generated by
  $\mathcal B$ and all the null-sets.
\end{theorem}
See \cite[Th. 2.20]{rudin:1987-3rd}.

\section{Measurable functions}
\label{sec:measurable-functions}



\section{Integrals}
\label{sec:integrals}

\bibliographystyle{amsplain}
\bibliography{tutto}

\end{document}

\usepackage{macros}

\title{P2021-0}
\author{Giovanni Pistone}
\date{DRAFT \today}

\begin{document}

\maketitle

This is a list of relevant theorems with references.

\section{Probability}
\label{sec:probability}

\section{Integrals}
\label{sec:integrals}

\bibliographystyle{amsplain}
\bibliography{tutto}

\end{document}
